\documentclass{article}
\usepackage{amsmath}
\usepackage{amsfonts}
\usepackage{mathtools}

\author{Salvador Castagnino \\ scastagnino@itba.edu.ar}
\date{} 
\title{Chapter 1 - Basic Measure Theory}

\begin{document}

\maketitle 

\section*{Exercise Solutions}

\begin{exercise}\textbf{Exercise 1.1.3}

\end{exericse}

\bigbreak

\begin{exercise}\textbf{Exercise 1.3.1}

\end{exericse}

\bigbreak

\begin{exercise}\textbf{Exercise 1.3.3}

\end{exericse}

\bigbreak

\begin{exercise}\textbf{Exercise 1.4.1}
	The \textit{only if} assertion is more than clear, we procede with the \textit{if} one. We base the proof on the observation that if $h$ is an even function then $x \in h^{-1}(A)$ implies $-x \in h^{-1}(A)$ for any $x \in \mathbb{R}$ and any $A \subset \mathbb{R}$. With this in mind, it suffices to show that $\mathcal{A} = \{ A \in \mathcal{B}(\mathbb{R}) : x \in A \implies -x \in A\} \subset \sigma(f)$, this is what we will do.

	More specific, we will show that for all $B \in \mathcal{B(\mathbb{R})}$ with $B \subset \mathbb{R}_{\geq 0}$, $B \cup -B \in \sigma(f)$. To do this observe that $\mathcal{C} = \{ (-\infty, -a] \cup [a, \infty): a \in \mathbb{R}_{\geq 0}\} \subset \sigma(f)$ and that by exchanging the generator with the trace we get
	\[
	\begin{align}
		\mathcal{B}(\mathbb{R})\vert_{\mathbb{R}_{\geq 0}} &= \sigma(\{[a, \infty) : a \in \mathbb{R}\})\vert_{\mathbb{R}_{\geq 0}} \\
								   &= \sigma(\{[a, \infty) : a \in \mathbb{R}_{\geq 0}\}) \\
								   &= \sigma(\mathcal{C})\vert_{\mathbb{R}_{\geq 0}}
	\end{align}
	\]
	The remaining steps are simple and left to the reader.

\end{exericse}

\bigbreak

\begin{exercise}\textbf{Exercise 1.4.2}
	Let $f$ be as in \textbf{Ex 1.4.1} and let $(\mathbb{R}, \sigma(f), \lambda)$ be our measure space, we define the function $g: \mathbb{R} \rightarrow \mathbb{R}$ such that
	\[
		g(x) = 
		\begin{cases*}
			2 & if $x=1$ \\
			-2 & if $x=-1$ \\
			|x| & otherwise
		\end{cases*}
	\]
	Now, $f$ and $g$ only differ in $\{1,-1\}$ which clearly has null measure. However, we have that $1 \in g^{-1}(2)$ and $-1 \notin g^{-1}(2)$ which implies that the set $g^{-1}(2)$ cannot be measurable (this assertion can be deduced from the solution of \textbf{Ex 1.4.1}) which conclues the proof.

\end{exericse}

\bigbreak

\begin{exercise}\textbf{Exercise 1.4.3}
	The differentiability of $f$ implies it's continuity which in turn implies it's measurability. Now define the sequence of functions
	\[
		f_n (c)= \frac{f(c+\frac{1}{n}) - f(c)}{(c + \frac{1}{n}) - c}
	\]
	It is easy to verify that these are measurable functions and given the existance of the limit of the difference quotient of $f$ we have $f'(c) = \limsup_{n\rightarrow\infty} f_n(c)$ for all $c \in \mathbb{R}$ which implies the measurability of $f'$ and concludes the proof.


\end{exericse}

\bigbreak

\begin{exercise}\textbf{Exercise 1.4.5}

\end{exericse}

\section*{Useful Properties}


\end{document}
