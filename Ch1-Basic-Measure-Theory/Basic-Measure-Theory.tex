\documentclass{article}
\usepackage{amsmath}
\usepackage{amsfonts}
\usepackage{mathtools}

\author{Salvador Castagnino \\ scastagnino@itba.edu.ar}
\date{} 
\title{Chapter 1 - Basic Measure Theory}

\begin{document}

\maketitle 

\section*{Exercise Solutions}

\begin{exercise}\textbf{Exercise 1.1.3}
	Let $\epsilon, \delta> 0$ we define 
	\[
		U_{\epsilon,\delta} = \{x \in \Omega_1 : \exists\ y, z \in B(x,\delta)\ \text{with}\ d_2(f(y),f(z)) > \epsilon \}
	\]
	Let's see that these sets are open for all $\epsilon$ and $\delta$ and that $U_f$ can be generated from them.
	
	Let $\epsilon, \delta > 0$ we take $x \in U_{\epsilon, \delta}$ and define $m = \delta - \max\{d_1(x,y),d_1(x,z)\}$, we want to see that $B(x,m) \subset U_{\epsilon, \delta}$. If $w \in B(x,m)$ then
	\[	
		d_1(w,y) \leq d_1(w,x) + d_1(x,y) < m + d_1(x,y) \leq \delta
	\]
	This proves that $y \in B(w,\delta)$ and the same reasoning can be applied to prove that $z \in B(w,\delta)$. Combining both statementes we get that $w \in U_{\epsilon, \delta}.
	
	Now, observe that $x \in \Omega_1$ is a point of discontinuity of $f$ if and only if there exists an $\epsilon > 0$ such that for all $\delta > 0$ we have that $f(B(x,\delta)) \not\subset B(f(x),\epsilon)$. Expressing this with set operations we get
	\[
		U_f = \bigcup_{n=1}^{\infty} \bigcap_{m=1}^{\infty} U_{\frac{1}{n},\frac{1}{m}}
	\]
	This proves what we wanted.

\end{exericse}

\bigbreak

\begin{exercise}\textbf{Exercise 1.3.1}

\end{exericse}

\bigbreak

\begin{exercise}\textbf{Exercise 1.3.3}

\end{exericse}

\bigbreak

\begin{exercise}\textbf{Exercise 1.4.1}
	The \textit{only if} assertion is more than clear, we procede with the \textit{if} one. We base the proof on the observation that if $h$ is an even function then $x \in h^{-1}(A)$ implies $-x \in h^{-1}(A)$ for any $x \in \mathbb{R}$ and any $A \subset \mathbb{R}$. With this in mind, it suffices to show that $\mathcal{A} = \{ A \in \mathcal{B}(\mathbb{R}) : x \in A \implies -x \in A\} \subset \sigma(f)$, this is what we will do.

	More specific, we will show that for all $B \in \mathcal{B(\mathbb{R})}$ with $B \subset \mathbb{R}_{\geq 0}$, $B \cup -B \in \sigma(f)$. To do this observe that $\mathcal{C} = \{ (-\infty, -a] \cup [a, \infty): a \in \mathbb{R}_{\geq 0}\} \subset \sigma(f)$ and that by exchanging the generator with the trace we get
	\[
	\begin{align}
		\mathcal{B}(\mathbb{R})\vert_{\mathbb{R}_{\geq 0}} &= \sigma(\{[a, \infty) : a \in \mathbb{R}\})\vert_{\mathbb{R}_{\geq 0}} \\
								   &= \sigma(\{[a, \infty) : a \in \mathbb{R}_{\geq 0}\}) \\
								   &= \sigma(\mathcal{C})\vert_{\mathbb{R}_{\geq 0}}
	\end{align}
	\]
	The remaining steps are simple and left to the reader.

\end{exericse}

\bigbreak

\begin{exercise}\textbf{Exercise 1.4.2}
	Let $f$ be as in \textbf{Ex 1.4.1} and let $(\mathbb{R}, \sigma(f), \lambda)$ be our measure space, we define the function $g: \mathbb{R} \rightarrow \mathbb{R}$ such that
	\[
		g(x) = 
		\begin{cases*}
			2 & if $x=1$ \\
			-2 & if $x=-1$ \\
			|x| & otherwise
		\end{cases*}
	\]
	Now, $f$ and $g$ only differ in $\{1,-1\}$ which clearly has null measure. However, we have that $1 \in g^{-1}(2)$ and $-1 \notin g^{-1}(2)$ which implies that the set $g^{-1}(2)$ cannot be measurable (this assertion can be deduced from the solution of \textbf{Ex 1.4.1}) which conclues the proof.

\end{exericse}

\bigbreak

\begin{exercise}\textbf{Exercise 1.4.3}
	The differentiability of $f$ implies it's continuity which in turn implies it's measurability. Now define the sequence of functions
	\[
		f_n (c)= \frac{f(c+\frac{1}{n}) - f(c)}{(c + \frac{1}{n}) - c}
	\]
	It is easy to verify that these are measurable functions and given the existance of the limit of the difference quotient of $f$ we have $f'(c) = \limsup_{n\rightarrow\infty} f_n(c)$ for all $c \in \mathbb{R}$ which implies the measurability of $f'$ and concludes the proof.


\end{exericse}

\bigbreak

\begin{exercise}\textbf{Exercise 1.4.5}

\end{exericse}

\bigbreak

\begin{exercise}\textbf{Exercise 1.5.2}
	Let's start by constructing a random variable $X$ as shown in \textbf{Thm 1.104}, we define the random vector $Z=(X,-X)$ which, as we'll see, is not normal. We have
	\[
		F_Z(x_1,x_2) = P[-x_2 \ge X \ge x_1]= \frac{1}{\sqrt{2 \pi \sigma^2}} \int_{x_1}^{-x_2} \exp\left\{-\frac{(t-\mu)^2}{2\sigma^2}\right\} dt
	\]
	Then it is clear that $F_Y(0,0)=0$, however that's not the case with $F_X(0,0)$ which concludes the proof.

\end{exericse}

\bigbreak

\begin{exercise}\textbf{Exercise 1.5.3}
	(i) Let $X \sim \mathcal{N}_{\mu,\sigma^2}$ then we have the \textit{normal distribution} as measure with continous density function $f$. We define $\varphi:\mathbb{R} \rightarrow \mathbb{R}$ to be the function $\varphi (x) = ax+b$ with $a \neq 0$ which is clearly a continously differentiale bijection.

	Now, \textbf{Thm 1.101} states the density function of $\mathcal{N}_{\mu, \sigma^2} \circ \varphi^{-1}$ to be
	\[
		f_{\varphi}(x) = \frac{f(\frac{x-b}{a})}{|a|}=\frac{1}{\sqrt{2\pi(\sigma a)^2}}
		\exp\left\{ - \frac{(x-(a\mu +b))^2}{2(\sigma a)^2} \right\}
	\]
	Given the equality of \textit{distribution functions} we have that $\mathcal{N}_{\mu,\sigma^2} \circ \varphi^{-1} = \mathcal{N}_{a\mu +b,(\sigma a)^2}$ and we know that $aX+b \sim \mathcal{N}_{\mu,\sigma^2} \circ \varphi^{-1}$ which concludes the proof.

	\bigbreak

	(ii) Let $X \sim \exp_\theta$ then we have the \textit{exponential distribution} as measure with continous density function $f$. We define $\varphi : \mathbb{R} \rightarrow \mathbb{R}$ to be the function $\varphi(x) = ax$ with $a > 0$ which is clearly a continously differentiable bijection. 

	Now, \textbf{Thm 1.101} states the density function of $\exp_\theta \circ \varphi^{-1}$ to be
	\[
		f_{\varphi}(x) = \frac{f(\frac{x}{a})}{a}=\frac{\theta}{a} \exp \left\{ - \frac{\theta}{a} x \right\}
	\]
	Given the equality of \textit{distribution functions} we have that $\exp_\theta \circ \varphi^{-1} = \exp_{\frac{\theta}{a}$ and we know that $aX \sim \exp_\theta \circ \varphi^{-1}$ which concludes the proof.
\end{exericse}

\bigbreak

\begin{exercise}\textbf{Exercise 1.5.4}
	We start with the \textit{only if} statement. It is easy to deduce the monotonicity of $F$ from that of the measure, we prove it's right continuity. Let $\{(x_n,y_n)\}_{n \in \mathbb{N}}$ be a sequence in $\mathbb{R}^2$ which converges to $(x,y)$, observe that if the sequence is decreasing we can directly apply the continuity of the measure. However if the sequence isn't decresing, this can't be done directly. 

	With this in mind our problem becomes to find a decreasing sequence such that it's continuity implies that of the original sequence. Define $N_m \in \mathbb{N}$ such that for all $n \ge N_m$ we have that $\| (x_n,y_n)-(x,y) \| < \frac{1}{m}$ and let $a_m = \sup_{n \ge N_m} x_n$ and $b_m = \sup_{n \ge N_m} y_n$.

	Firstly we have that $(x_n,y_n) \le (a_m,b_m)$ for all $n \ge N_m$ and thus 
	\[
		(-\infty, (x_n,y_n)] \subset (-\infty, (a_n,b_n)]
	\]
	Secondly, we have that $x \le a_m \le x+\frac{1}{m}$ and $y \le b_m \le y+\frac{1}{m}$ which implies that 
	\[
		\bigcap_{m=1}^{\infty} (-\infty, (a_m,b_m)] = (-\infty,(x,y)]
	\]
	These two last statements prove what we wanted. Point \textit{(ii)} can be proved using the same idea. Observe that we talk about the \textit{right continuity} and not the \textit{left continuity} given that a union of sets of the form $(-\infty, x_n]$ may not contain the point $\sup_{n \ge 1} x_n$ while it's intersection will always contain the point $\inf_{n \ge 1} x_n$.

	Finaly, point \textit(iii) can be easily proven by provinding a convenient partition for sets of the form $(-\infty, x]$ with $x \in \mathbb{R}^2$.
\end{exericse}

\bigbreak

\begin{exercise}\textbf{Exercise 1.5.5}

\end{exericse}

\bigbreak

\section*{Useful Properties}


\end{document}
